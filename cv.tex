\documentclass[a4paper, 10pt]{article}

\usepackage[italian]{babel}
\usepackage[utf8]{inputenc}
\usepackage{fouriernc}

\title{Curriculum Vitae}
\author{Emanuele Galardo}
\date{\today}

\setcounter{tocdepth}{1}

\begin{document}
\maketitle

%\tableofcontents

\section{Informazioni personali}
\begin{tabular}{l l}
    \textbf{Nome:}                    & Emanuele                   \\
    \textbf{Cognome:}                 & Galardo                    \\
    \textbf{Nazionalità:}             & Italiana                   \\
    \textbf{Telefono:}                & 328 305 3799               \\
    \textbf{Email:}                   & me@galardo.net             \\
    \textbf{Data e Luogo di Nascita:} & 30 Novembre 2002 - Crotone \\
    \textbf{Web:}                     & galardo.net                \\
\end{tabular}

%\section{Posizione Attuale}

\section{Istruzione e Formazione}

\paragraph{\textbf{Laurea Triennale in Informatica}}
Attualmente sono un laureando in Informatica presso l'Università della Calabria.

\paragraph{\textbf{Diploma Tecnico Industriale}}
Nel 2021 ho conseguito il diploma presso l'Istituto Tecnico Industriale "Guido Donegani" di Crotone con la votazione di 100/100 nell'articolazione Informatica e Telecomunicazione, ramo Informatica

\subsection{Altri titoli e certificati}

\paragraph{\textbf{Cambridge FCE}}
Certificato di conoscenza della lingua inglese di livello B2 rilasciato dall'Università di Cambridge conseguito a Marzo 2021 con un punteggio di 169/190.

\paragraph{\textbf{Cambridge PET}}
Certificato di conoscenza della lingua inglese di livello B1 rilasciato dall'Università di Cambridge conseguito a Giugno 2020 con un punteggio di 157/170.

\subsection{Scuole di formazione e stage}

\paragraph{\textbf{CyberChallenge 2023}}
Partecipazione al programma di addestramento in sicurezza informatica CyberChallenge 2023 organizzato dalla Cyber Security National Lab.

\section{Premi e Riconoscimenti}

\paragraph{\textbf{Best Students Award (2nd yr, class 2021-2022)}}
Premio assegnato dal DeMaCS insieme al CdL in Informatica ai migliori studenti del secondo anno in Informatica della coorte 2021/2022. Il premio è assegnato agli studenti che hanno ottenuto una media sui migliori 80 CFU superiore a 28/30.

\paragraph{\textbf{Best Students Award (1st yr, class 2021-2022)}}
Premio assegnato dal DeMaCS insieme al CdL in Informatica ai migliori studenti del primo anno in Informatica della coorte 2021/2022. Il premio è assegnato agli studenti che hanno ottenuto una media sui migliori 40 CFU superiore a 28/30.

\paragraph{\textbf{Esonero per merito 2023/2024}}
Esonero parziale dalla tassa universitaria per merito accademico per l'anno accademico 2023/2024, per aver ottenuto tutti i CFU previsti per l'anno accademico 2022/2023.

\paragraph{\textbf{Esonero per merito 2022/2023}}
Esonero parziale dalla tassa universitaria per merito accademico per l'anno accademico 2022/2023, per aver ottenuto tutti i CFU previsti per l'anno accademico 2021/2022.

\section{Convegni e Seminari}

\paragraph{\textbf{Videogiochi e Intelligenza Artificiale}}
Seminario tenuto il 2 Maggio 2024 dal Corso di Laurea in Informatica dell'Università della Calabria. Sono stato relatore discutendo di "Robot e Giochi su Dispositivi Mobili".

%\section{Esperienze Lavorative}

\section{Patenti}

\begin{itemize}
    \item Patente B
    \item Patente AM
    \item APR Operazioni Non Critiche
\end{itemize}

\end{document}